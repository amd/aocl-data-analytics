% Copyright (c) 2016, The Science and Technology Facilities Council (STFC)
% All rights reserved.
% setup the example problem

Consider fitting the function $y(t) = x_1e^{x_2 t}$ to data $(\bm{t}, \bm{y})$
using a non-linear least squares fit.\\
The residual function is given by
$$
   r_i(\vx)  = x_1 e^{x_2 t_i} - y_i.
$$
We can calculate the Jacobian and Hessian of the residual as
$$
   \nabla r_i(\vx) = \left(\begin{array}{cc}
      e^{x_2 t_i} &
      t_i x_1 e^{x_2 t_i}
      \end{array}\right),
$$
$$
   \nabla^2 r_i(\vx) = \left(\begin{array}{cc}
      0                 & t_i e^{x_2 t_i}    \\
      t_i e^{x_2 t_i}     & x_1 t_i^2 e^{x_2 t_i}
   \end{array}\right).
$$

For some data points, $y_i$, $t_i$, $(i = 1,\dots,m)$ the user must return
$$  \vr(\vx) = \begin{bmatrix}
      r_1(\vx)\\
      \vdots \\
      r_m(\vx)
    \end{bmatrix}, \quad   \vJ(\vx) =
    \begin{bmatrix}
      \nabla r_1(\vx) \\
      \vdots \\
      \nabla r_m(\vx) \\
    \end{bmatrix}, \quad
    Hf(\vx) =
    \sum_{i=1}^m
    (\vr)_i \nabla^2 r_i(\vx),
$$
where, in the case of the Hessian, $(\vr)_i$ is the $i$th component of a residual vector passed to the user.

Given the data
\begin{center}
   \begin{tabular}{l|*{5}{r}}
      $i$   & 1 & 2 & 3  & 4  & 5 \\
      \hline
      $t_i$ & 1 & 2 & 4  & 5  & 8 \\
      $y_i$ & 3 & 4 & 6 & 11 & 20
   \end{tabular}
\end{center}
and initial guess $\vx = (2.5, 0.25)$, the following code performs the fit (with no
weightings, i.e., $\vW = \vI$).

%%% Local Variables:
%%% mode: latex
%%% TeX-master: "nlls_fortran"
%%% End: